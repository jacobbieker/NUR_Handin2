\section{Part 1}

The shared code for this part, as well as the random generator for all the parts is here:

\lstinputlisting{one.py}

\section{Part 1.a Normally Distributed pseudo-random numbers}

The file of the functions used for this exercise is:

\lstinputlisting{one_a.py}

My script produces the following plots, see Fig. \ref{fig:xi_xi}, Fig. \ref{fig:thousand}, and Fig. \ref{fig:million}.
These plots suggest the random number generator is sufficiently random and is not biased towards any numbers.

\begin{figure}[h!]
  \centering
  \includegraphics[width=0.9\linewidth]{./plots/Xi_Xi_1.png}
  \caption{The results of the first 1000 numbers of the random generator. }
  \label{fig:xi_xi}
\end{figure}

\begin{figure}[h!]
  \centering
  \includegraphics[width=0.9\linewidth]{./plots/Index_Xi_1.png}
  \caption{First 1000 random numbers vs index.}
  \label{fig:thousand}
\end{figure}

\begin{figure}[h!]
  \centering
  \includegraphics[width=0.9\linewidth]{./plots/1000000_rand.png}
  \caption{1,000,000 random numbers plotted in 20 bins 0.05 wide.}
  \label{fig:million}
\end{figure}


\section{Part 1.b Box-Muller Method}

The file of the functions used for this exercise is:

\lstinputlisting{one_b.py}

My script produces the following result, see Fig. \ref{fig:boxmuller}. As can be seen, the Box-Muller implementation seems
to create a Gaussian distribution.

\begin{figure}[h!]
  \centering
  \includegraphics[width=0.9\linewidth]{./plots/box_gauss.png}
  \caption{Box-Muller method.}
  \label{fig:boxmuller}
\end{figure}


\section{Part 1.c KS-test}

The file of the functions used for this exercise is:

\lstinputlisting{one_c.py}

The result of this function is given by the Figures \ref{fig:kstest}, and \ref{fig:kstestP}.

\begin{figure}[h!]
  \centering
  \includegraphics[width=0.9\linewidth]{./plots/KStest.png}
  \caption{KS Test.}
  \label{fig:kstest}
\end{figure}


\begin{figure}[h!]
  \centering
  \includegraphics[width=0.9\linewidth]{./plots/KStest_pvalue.png}
  \caption{KS Test P-values.}
  \label{fig:kstestP}
\end{figure}

This seems to show that the KS tests are fairly similar in their results, and their results seem to
suggest that the Box-Muller implementation is consistent with a Gaussian distribution.


\section{Part 1.d Kuiper Test}

The file of the functions used for this exercise is:

\lstinputlisting{one_d.py}

My script produces the following result, see Fig. \ref{fig:kuiperTest}

\begin{figure}[h!]
  \centering
  \includegraphics[width=0.9\linewidth]{./plots/KuiperTest.png}
  \caption{Kuiper Test implementation.}
  \label{fig:kuiperTest}
\end{figure}

This seems to show that the probabilities derived from my Kuiper Test differs somewhat from the Kuiper test in Astropy, although the cause for that is unknown. At
the same time, the value for V is similar between both the Astropy version and my own.
It also shows that my the Box-Muller implementation seems to still be consistent with a Gaussian Distribution.


\section{Part 1.e 10 Sets of Numbers}

The file of the functions used for this exercise is:

\lstinputlisting{one_e.py}

My script produces the following result, see Fig. \ref{fig:10_sets}.


\begin{figure}[h!]
  \centering
  \includegraphics[width=0.9\linewidth]{./plots/RandNumKS.png}
  \caption{Results of KS Test on 10 sets of numbers.}
  \label{fig:10_sets}
\end{figure}

\begin{figure}[h!]
  \centering
  \includegraphics[width=0.9\linewidth]{./plots/RandNumKS_sci.png}
  \caption{Results of KS Test on 10 sets of numbers with Scipy norm.cdf.}
  \label{fig:10_sets_sci}
\end{figure}

From these results, we can see some of the random number sets are consistent with Gaussians for small amounts of numbers,
but when using $10^5$ points, none of the sets of random numbers seems consistent. Overall, the closest seems to be
set 3 from the p-values. When lookng at the results using Scipy's norm.cdf as comparison, there seems to also be set 5 that
close to a Gaussian the most times on the graph compared to other ones, but is still not consistent at $10^5$ points. Therefore,
I conclude that sets 5 and 3 are the most consistent with a Gaussian distribution when taking in large numbers of points, sets
1, 9, and 6 are consistent with a Gaussian at specific numbers of points, but none are consistent over all the range in numbers of points.

