\section{Part 2. Gaussian Random Field}

The file of the functions used for this exercise is:

\lstinputlisting{part_two.py}

My script produces the following results, showing the Gaussian random fields for n = -1, -2, and -3, in Figures \ref{fig:gauss1}, \ref{fig:gauss2}, and \ref{fig:gauss3}, respecitvely.
As I chose the size to be in Mpc, the minimum size that these plots show is 1 Mpc. This affects the
maximum size in conjunction with $n$, as $n$ influences the structure formation on different scales. So, for example,
in Fig. \ref{fig:gauss3}, there are over and underdensities on the scale of a few hundred Mpc, while for
$n = -1$, the over and underdensities are on the order of one to a couple of Mpc in size. If I had chosen a physical size
of 1 pixel is 1 kpc instead, the same would hold true, with the largest over and underdensities for $n = -3$ being a few hundred kpc
in size, while for a larger $n$, $n = -1$, they would be at much smaller scales.

Additionally, as $n$ increases from -3 to -1, $k$'s absolute value goes up, suggesting that the minimum and maximum $k$
values get larger. For $n = -3$, the maximum over or underdensity has a $k$ value of around 0.003, while for $n = -1$, that value
goes up by nearly an order of magnitude to 0.025. The minimum $k$ value between those two $n$ values similar changes, going from 0.0005
to 0.005.

The larger $k$ values for larger $n$ valules also corresponds to smaller, more detailed structures in the density field, while
lower $k$ values correspond to much larger in space fluctuations.


\begin{figure}[h!]
  \centering
  \includegraphics[width=0.9\linewidth]{./plots/GaussianField-1.png}
  \caption{Gauss Random Field for n = -1}
  \label{fig:gauss1}
\end{figure}

\begin{figure}[h!]
  \centering
  \includegraphics[width=0.9\linewidth]{./plots/GaussianField-2.png}
  \caption{Gauss Random Field for n = -2}
  \label{fig:gauss2}
\end{figure}

\begin{figure}[h!]
  \centering
  \includegraphics[width=0.9\linewidth]{./plots/GaussianField-3.png}
  \caption{Gauss Random Field for n = -3}
  \label{fig:gauss3}
\end{figure}

