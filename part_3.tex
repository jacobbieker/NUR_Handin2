\section{Part 3. Linear Structure Growth}

The file of the functions used for this exercise is:

\lstinputlisting{part_three.py}

The result of this function is shown in \ref{fig:lgf}:

\begin{figure}[h!]
  \centering
  \includegraphics[width=0.9\linewidth]{./plots/growth_factors.png}
  \caption{Linear Growth Factors for different cases.}
  \label{fig:lgf}
\end{figure}

As can be seen from the plot, the numerical solution is close the analtical solution, but not exact.
The runge-kutta 4th order method was chosen for its ease of implementation and relative accuracy.
One issue that I was not able to figure out was for the third case, the analytical and numerical solutions differ by
quite a bit. This is unexpected, since the other two cases, it does match fairly closely.

I did not have time to determine why this is the case and fix it.