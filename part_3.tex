\section{Part 3.a Likelihood maximization}

For this part, I could not calculate the log-likelihood for the distributions. At the same time, to continue on,
the satellite distribution for the given mass bin can be shown to change as the mass increases. This is shown in Figures
\ref{fig:m11}, \ref{fig:m12}, \ref{fig:m13}, \ref{fig:m14}, and \ref{fig:m15}. To complete this task, what I should have
done would have been to construct, based on the given $n(x)$, the log-likelihood for where most of the satellites would
be located in $x$ dependent on $a$,$b$,and $c$. The maximum likelihood would be where the peak of the satellite distribution
is located on these figures.

The file of the functions used for this exercise is:

\lstinputlisting{three_a.py}

The result of this function is given by:

%\lstinputlisting{3a.txt}

\begin{figure}[h!]
  \centering
  \includegraphics[width=0.9\linewidth]{./plots/satgals_m11.png}
  \caption{Radial distribution of satellites in m11.txt}
  \label{fig:m11}
\end{figure}

\begin{figure}[h!]
  \centering
  \includegraphics[width=0.9\linewidth]{./plots/satgals_m12.png}
  \caption{Radial distribution of satellites in m12.txt}
  \label{fig:m12}
\end{figure}

\begin{figure}[h!]
  \centering
  \includegraphics[width=0.9\linewidth]{./plots/satgals_m13.png}
  \caption{Radial distribution of satellites in m13.txt}
  \label{fig:m13}
\end{figure}

\begin{figure}[h!]
  \centering
  \includegraphics[width=0.9\linewidth]{./plots/satgals_m14.png}
  \caption{Radial distribution of satellites in m14.txt}
  \label{fig:m14}
\end{figure}

\begin{figure}[h!]
  \centering
  \includegraphics[width=0.9\linewidth]{./plots/satgals_m15.png}
  \caption{Radial distribution of satellites in m15.txt}
  \label{fig:m15}
\end{figure}

\section{Part 3.b Function Fitting}

The file of the functions used for this exercise is:

\lstinputlisting{three_b.py}

As in part a), for this part I did not successfully create an interpolator for a,b,and c based off the halo mass bin.