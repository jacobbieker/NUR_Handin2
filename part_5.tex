\section{Part 5.a Nearest Grid Point Method}

The file of the functions used for this exercise is:

\lstinputlisting{part_five.py}

While I ran out of time to finish this part, here is the methodology I was attempting to make.
I did generate the grid, as well as the random points, and started on the Nearest Grid Point method. The idea was to
go to thebox of 8 points surrounding the mass that is being added, calculate the distance to each of the grid points, and
add all the mass to the closest of the 8 points. While not the most efficient, it would have resulted in the mass being
distributed properly.

\section{Part 5.c Cloud In Cell Method}

For the Cloud In Cell Method, while again, I ran out of time to implement it, to do so I would have done as follows.
I would have gotten the closest 8 points making up the box around the mass, in the same way as for the Nearest Grid Point method.
Once having those points, the next step is calculating how much of the particle is the closest to each grid point, and assigning that
portion of the particle's mass to the grid point it is closest to.