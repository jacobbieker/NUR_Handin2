\section{Part 5.a Nearest Grid Point Method}

The file of the functions used for this exercise is:

\lstinputlisting{five_a.py}

The result of this function is given by:

\begin{figure}[h!]
  \centering
  \includegraphics[width=0.9\linewidth]{./plots/interpolation.png}
  \caption{The interpolation function in action.}
  \label{fig:54}
\end{figure}

\begin{figure}[h!]
  \centering
  \includegraphics[width=0.9\linewidth]{./plots/interpolation.png}
  \caption{The interpolation function in action.}
  \label{fig:59}
\end{figure}

\begin{figure}[h!]
  \centering
  \includegraphics[width=0.9\linewidth]{./plots/interpolation.png}
  \caption{The interpolation function in action.}
  \label{fig:511}
\end{figure}

\begin{figure}[h!]
  \centering
  \includegraphics[width=0.9\linewidth]{./plots/interpolation.png}
  \caption{The interpolation function in action.}
  \label{fig:514}
\end{figure}


\section{Part 2.b Interpolation}

The file of the functions used for this exercise is:

\lstinputlisting{two_b.py}

My script produces the following result, see Fig. \ref{fig:interp}. This shows the differences between the linear
interpolation and cubic spline. Both are essentially the same for this data, except between 0.1 and 1. In comparison to
the actual values, the interpolated values underestimate for all values between $x$ = 1 and $x$ = 5.

\begin{figure}[h!]
  \centering
  \includegraphics[width=0.9\linewidth]{./plots/interpolation.png}
  \caption{The interpolation function in action.}
  \label{fig:interp}
\end{figure}
