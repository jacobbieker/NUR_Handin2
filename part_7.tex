\section{Part 2.a Numerical Integrator}

The file of the functions used for this exercise is:

\lstinputlisting{two_a.py}

The result of this function is given by:

\lstinputlisting{2a.txt}

\section{Part 2.b Interpolation}

The file of the functions used for this exercise is:

\lstinputlisting{two_b.py}

My script produces the following result, see Fig. \ref{fig:interp}. This shows the differences between the linear
interpolation and cubic spline. Both are essentially the same for this data, except between 0.1 and 1. In comparison to
the actual values, the interpolated values underestimate for all values between $x$ = 1 and $x$ = 5.

\begin{figure}[h!]
  \centering
  \includegraphics[width=0.9\linewidth]{./plots/interpolation.png}
  \caption{The interpolation function in action.}
  \label{fig:interp}
\end{figure}

\section{Part 2.c Numerical Derivative}

The file of the functions used for this exercise is:

\lstinputlisting{two_c.py}

The result of this function is given by:

\lstinputlisting{2c.txt}

\section{Part 2.d Random Sampling}

The file of the functions used for this exercise is:

\lstinputlisting{two_d.py}

My script produces the following result, see Fig. \ref{fig:rand_sample}

\begin{figure}[h!]
  \centering
  \includegraphics[width=0.9\linewidth]{./plots/random_sample.png}
  \caption{The results of the random sample with rejection sampling.}
  \label{fig:rand_sample}
\end{figure}

The positions in (r, $\phi$, $\theta$) are given by:

\lstinputlisting{2d.txt}

\section{Part 2.e 1000 Haloes}

The file of the functions used for this exercise is:

\lstinputlisting{two_e.py}

My script produces the following result, see Fig. \ref{fig:1000_haloes}. As can be seen, the generated galaxies mostly match
the $N(x) = n(x)4\pi x^2$ distribution.

\begin{figure}[h!]
  \centering
  \includegraphics[width=0.9\linewidth]{./plots/1000_haloes.png}
  \caption{The results of 1000 haloes each with 100 satellites. }
  \label{fig:1000_haloes}
\end{figure}

\section{Part 2.f Root Finding}

The file of the functions used for this exercise is:

\lstinputlisting{two_f.py}

The result of this function is given by:

\lstinputlisting{2f.txt}

\section{Part 2.g Percentiles and Poisson}

The file of the functions used for this exercise is:

\lstinputlisting{two_g.py}

The median, 16th, and 84th percentiles for the radial bin with the largest number of falaxies is given in:

\lstinputlisting{2g.txt}

My script produces the following result, see Fig. \ref{fig:hist_poisson}. As can be seen, histogram fairly closely follows
the Poisson distribution with $\lambda$ equal to the mean number of galaxies in this radial bin.

\begin{figure}[h!]
  \centering
  \includegraphics[width=0.9\linewidth]{./plots/hist_poisson.png}
  \caption{The results of 1000 haloes each with 100 satellites. }
  \label{fig:hist_poisson}
\end{figure}

\section{Part 2.h 3D interpolator}

The file of the functions used for this exercise is:

\lstinputlisting{two_h.py}

While no output was requested, I did test the 3D interpolator, with the a,b,and c values generated earlier.
The error is quite large for these, which I believe is a result of the linear interpolation used. Cubic splines took too
long to run, and were expensive to create.

\lstinputlisting{2h.txt}



